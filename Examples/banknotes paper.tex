Which dataset was used in the following text?

Deploying reliable ML models for banknote recognition requires images of banknotes in real assistive scenarios, i.e.
varying background, illumination, blur, orientation, occlusion and focus conditions. Although, high quality scans of
multiple currencies are available in banknote collectors’ websites such as [3, 6 ], abundant data of real usage scenarios
for assistive applications is very limited. In consequence, we design our data collection process with this accessibility
application in mind.

BankNote-Net: Open dataset for assistive universal currency recognition 3
Fig. 1. Our approach: we use a convolutional neural network (MobileNet V2) as an encoder for our assistive banknote images. Using
supervised contrastive learning, the models learns an highly-compressed and descriptive embedding for each image. We use these
embeddings for downstream tasks, such as recognition of Yen banknotes, as shown in the figure. Our final dataset is composed of the
learned embeddings for 17 currencies and 224 classes. "Dense Classifier" corresponds to a fully-connected neural network, and "GAP"
corresponds to a Global Average Pooling layer.
The final data collection process consisted of the following steps:
(1) Currency selection: we selected the particular currencies to collect according to population and geographic
diversity, and according to areas with high number of users of the Microsoft Seeing AI app.
(2) Recruitment: for each specific currency, we on-boarded consenting volunteers or Amazon Mechanical Turk [2 ]
workers to perform data collection.
(3) Training: each worker or volunteer was provided with a brochure of requirements for the desired pictures along
with example pictures. This brochure is included in Appendix A.
(4) Data collection: each worker or volunteer captured and labelled up around 100 images for each orientation
(front or back) of each denomination of a specific currency. Each picture was captured with a square aspect
ratio to facilitate preprocessing. The images of each currency were captured exclusively by 1 or 2 volunteers
or workers located in the country where a specific currency is used; this decision was made to improve data
completeness across all denominations of a given currency.
Manuscript submitted to ACM


(5) Data validation and preparation: the collected data was validated manually by the authors to check for
inconsistencies in labelling or limited image diversity. Then, the file formats were standardized and all images
were resized to 224 x 224 pixels.
In total, 24,826 images were collected from 17 currencies. Figure 2a presents the distribution of collected images
across currencies, the distribution of images per class, and multiple examples of the collected images. The images span
112 denominations and 224 classes (each denomination maps to two classes, corresponding to the front and back faces
of the banknote). The mean number of images per class is 110, as shown in Figure 2. For banknotes that have more
than a one version in circulation, we defined each banknote version as a different class. Appendix B has a list of all
denominations and classes in our dataset.

The dataset consists